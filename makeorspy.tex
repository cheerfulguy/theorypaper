\title{\textsc{Make or Spy?}}
\author{
        \textsc{Abhishek Nagaraj} \\
         \small{\url{nagaraj@mit.edu}}
        %%\small{MIT Sloan School of Management}\\
        %%\small{50 Memorial Drive, E62-368,}\\
        %% \small{Cambridge, MA 02142}\\
}
\date{\today}

\documentclass[12pt]{article}

\usepackage{url}
\usepackage{nicefrac}
\usepackage{amsmath}
\usepackage{amssymb}

\begin{document}
\setlength{\parindent}{0pt}
\setlength{\parskip}{10pt}

    \newtheorem{theorem}{Theorem}[section]
    \newtheorem{lemma}[theorem]{Lemma}
    \newtheorem{proposition}[theorem]{Proposition}
    \newtheorem{corollary}[theorem]{Corollary}

    \newenvironment{proof}[1][Proof]{\begin{trivlist}
    \item[\hskip \labelsep {\bfseries #1}]}{\end{trivlist}}
    \newenvironment{definition}[1][Definition]{\begin{trivlist}
    \item[\hskip \labelsep {\bfseries #1}]}{\end{trivlist}}
    \newenvironment{example}[1][Example]{\begin{trivlist}
    \item[\hskip \labelsep {\bfseries #1}]}{\end{trivlist}}
    \newenvironment{remark}[1][Remark]{\begin{trivlist}
    \item[\hskip \labelsep {\bfseries #1}]}{\end{trivlist}}

    \newcommand{\qed}{\nobreak \ifvmode \relax \else
          \ifdim\lastskip<1.5em \hskip-\lastskip
          \hskip1.5em plus0em minus0.5em \fi \nobreak
          \vrule height0.75em width0.5em depth0.25em\fi}


\maketitle

\begin{abstract}
\end{abstract}

\section{Setup}

Upstream good $P$ (``platform'') and
downstream good $A$ (``application'') are supplied by two monopoly suppliers. Product $P$ can be consumed independently while product $A$ can
only be consumed in conjunction with $P$.  Fixed costs of entry
for $A$ are $C_A$ while those for $P$ are normalized to zero and all
marginal costs are zero. Consumers are indexed by $i$
($i \in \mathbb{R}^+$) and have valuations $V_P^i$ and  $V_A^i$ over the
two items. These are defined as $V_P^i = V_P - i -p$ and  $V_A^i =  V_A -
ki -(p+\rho)$ where $p$ is the price of $P$ and $\rho$ the price for $A$.
Each consumer consumes at most one unit of each good. 

In this setting I study the following problem:

What are equilibrium revenues $R_P$ and $R_A$ when the two producers play a sequential
pricing game where $p$ is announced in stage 1 and
conditional on entry $\rho$ in stage 2? 

%examples from real life? platform+app, OS+app, theme park + seller, 


\section{Solution}

The setup is completely described by three key parameters: \{$V_P$\}
characterizing $P$ and \{$V_A$, $k$\} characterizing $A$. 
With respect to $V_P$, I define four types of $A$ by splitting the
parameter space as follows:
\begin{center}

\begin{tabular}{ | l | c  r |}

\hline 

             & $V_P>\nicefrac{V_A}{k}$ & $V_P<\nicefrac{V_A}{k}$ \\ \hline
  $V_A>V_P$  & High Type & Good Type \\ 
   $V_A<V_P$ & Bad Type & Low Type \\ 
\hline
\end{tabular}

\end{center}


A solution for each of these four types is given below. 

\begin{enumerate}

\item \textbf{Bad Types} \\
In equilibrium, $R_A=0$ and $R_P=\nicefrac{V_P^2}{4}$
with $p=\nicefrac{V_p}{2}$ and $\rho=0$ with no entry from $A$.

\item \textbf{Good Types} \\
In equilibrium, $R_A=0$ and $R_P=\nicefrac{V_P^2}{4}$
with $p=\nicefrac{V_p}{2}$ and $\rho=0$ with no entry from $A$.



\end{enumerate}






In what follows I show that under certain conditions only
sufficiently ``specialized'' ideas (\emph{sufficiently} high or \emph{sufficiently}
low type) do not face hold-up. The nature of the problem is
not symmetric between the high and low types and is more acute for
low types. Because General ideas face hold-up at all times, a
well-known result in the literature on bundling, I will focus on
high/low types.  (insert a section on how general applications relates
to bundling, double marginalization etc)

Further, in the analysis both the platform and the developer have zero marginal
costs. The fixed cost for the Platform is assumed to be zero for
simplicity. 

\subsection{Timing}
The two players, the platform and the developer play a pricing game that is described as
follows: 
\begin{itemize}
\item \textbf{Period 1}: Platform specifies access price $p$. In extensions of
  the model the platform will be able to specify a number of
  contracting mechanisms including charging a royalty based on
  application sales but in the baseline model the platform makes
  no direct revenue from developers (\`a la Microsoft Windows).
\item  \textbf{Period 2}: Developer makes entry decision by incurring fixed cost
  of entry $C_E$ and sets price $\rho$ for the application
\item  \textbf{Period 3}: Platform decides whether it wants to imitate the application's idea at
  cost $C_I$. If it imitates the application then (a) the only
  available product on the market is the platform with the application
  baked in\footnote{why no mixed bundling?}. The developer is driven
  out of business (has zero revenue).\footnote{justification?}
\item  \textbf{Period 4}: Consumers make purchase decisions and payoffs are
  realized. 
\end{itemize}


\section{Solution}

Define $i^*$ such that $V_{i^*}^P=V_{i^*}^A$ i.e. the consumer who
has no marginal (positive or negative) valuation for the
application. Consumer $i^*$ exists only for high/low type of
applications and is unique. It is easy to see that
$i^*=\nicefrac{(V_P-V_A)}{(1-k)}$. Let $V^{i^*}_P =
V^{i^*}_A = V^{i^*}$

\subsection{Period 2: Specifying the Developer's problem}

In the section I derive the developer's demand $D(p,\rho)$.

Consumer $i$ demands the application if she satisfies the following two constraints:

\begin{equation}
\label{eq:ic}
V_A^i>V_P^i
\end{equation}

\begin{equation}
\label{eq:pc}
V_A^i>0
\end{equation}

I will call equation (\ref{eq:ic}) the incentive constraint(IC) and
equation (\ref{eq:pc}) the
participation constraint(PC). 

Rewriting, IC becomes: $$i(1-k)>p-(V_A-V_P)$$ and PC becomes $$i<\frac{V_A-(p+\rho)}{k}$$

These two constraints bind in different ways for high and low
types to generate different demand for each type as follows:

\begin{proposition}
\label{prop1}
In Period 2 for high types, the best response to a price $p$ set by the platform in
Period 1 is given by:\[ \rho^* = 
\begin{cases}
%%  \frac{V_A-V_P}{2} & \text{if $p < \frac{kV_P-V_A}{k-1}$} \\
%%  \frac{V_A-p}{2} & \text{if $p > V_P$} \\
%%  \text{solution} & \text{otherwise}

  \frac{V_A-V_P}{2} & \text{if $p < \frac{V_P(2k-1)-V_A}{2(k-1)}$} \\
  \frac{V_A-p}{2} & \text{if $p > \frac{2kV_P-V_A}{2k-1}$} \\
  (k-1)p-(kV_P-V_A) & \text{otherwise}


\end{cases} \]
\end{proposition}

\begin{proof}

Proof goes here

% To begin note that because in this case $k>1$, 
% the IC becomes: $$i<\frac{(V_A-V_P)-\rho}{(k-1)}$$ and PC
% remains $$i<\frac{V_A-(p+\rho)}{k}$$

% Therefore demand for the application 
% $$D(p,\rho)=Min\left[\frac{(V_A-V_P)-\rho}{(k-1)},
%   \frac{V_A-(p+\rho)}{k}\right]$$. 

% Define $\tilde{\rho}(p)=(k-1)p-(kV_P-V_A)$. Then we have the following
% demand function:\[ D(p,\rho) = 
% \begin{cases}
%   \frac{(V_A-V_P)-\rho}{(k-1)}  & \text{if $\rho>\tilde{\rho}(p)$ (IC binds)} \\
%   \frac{V_A-(p+\rho)}{k}   & \text{if $\rho<\tilde{\rho}(p)$ (PC binds)}
% \end{cases} \]

% The the developer's maximization problem is:
% $$\displaystyle \max_\rho \rho D(p,\rho)$$

% Now we use the condition that $0<\rho<(V_A-V_P)$ to obtain the result
% stated in Proposition \ref{prop1}. 

% \begin{align}
% \label{eq:cond1}
% 0<p<\frac{kV_P-V_A}{(k-1)}  & \implies  \tilde{\rho}<0 & \text{IC
%   binds always}\\
% \label{eq:cond2}
% V_A>p>V_P & \implies \tilde{\rho}>V_A-V_P  & \text{PC binds always}
% \end{align}

% Therefore when we are in the case specified by equation (\ref{eq:cond1}) the
% developer solves \[\rho^* = \displaystyle \max_\rho \left[ \rho \;
%   \frac{(V_A-V_P)-\rho}{(k-1)} \right] = \frac{V_A-V_P}{2}\] 

% And when we are in the case specified by equation (\ref{eq:cond2}) the
% developer solves \[\rho^* = \displaystyle \max_\rho \left[ \rho \;
%    \frac{V_A-(p+\rho)}{k}  \right] = \frac{V_A-p}{2}\] 

% Now consider the case when $\frac{kV_P-V_A}{k-1}<p<V_P$. We are
% looking for a $\rho$ subject to $p<\rho<V_A$ that maximizes
% revenue. 

% Now, when $p<\rho<\tilde{\rho}$ revenue is given by: 

% \begin{align*}
% \underline{\rho}^* = \displaystyle & \max_\rho \left[ \rho \; \frac{(V_A-p)-\rho}{k}
% \right]\\\\
% &ST: \;\; p<\rho<\tilde{\rho}\\
% \end{align*}

% The solution to the above problem is $\frac{V_A-p}{2}$ given that $\frac{k}{k-1}>\frac{V_A}{V_P}$\footnote{check}


% When $\tilde{\rho}<\rho<V_A$ revenue is given by: 

% \begin{align*}
% \bar{\rho}^* = \displaystyle & \max_\rho \left[ \rho \; \frac{(V_A-V_P)-\rho}{k}
% \right]\\\\
% &ST: \;\; \tilde{\rho}<\rho<V_P\\
% \text{revenue:}
% \end{align*}


\end{proof}

The intuition for the result in \ref{prop1} is as follows: 

When the platform's price is \emph{sufficiently low} the developer is able to
target his entire captive audience (all $i$ such that $V_A^i > V_P^i$)
and price like a monopolist over the consumer base. Note that the
optimal price does not depend upon the exact value of $p$. Therefore
effectively in this case every consumer who values the application
sufficiently higher than the platform buys the application in
equilibrium, while everyone else with $V_P^i>0$ buys just the platform.

When the platform's price is \emph{sufficiently high} only the
Participation Constraint binds. This means that buying just the
platform is not valuable for \emph{any} consumer, and \emph{everyone}
who buys the platform must also purchase the app. Formally, $V_P^i<0
\; \forall \; i$. In this case the platform is forgoing additional
userbase (by pricing them out) in order to get a larger share of the application's
revenue by increasing the platform's price.

\begin{proposition}
\label{prop2}
In Period 1, against high types the platform plays the following strategy:
Set equilibrium price in this period:\[ p^* = 
\begin{cases}

  \frac{V_A}{2} & \text{if ${V_P^2}<\nicefrac{V_A^2}{2k}$} \\
  \frac{V_P}{2} & \text{if ${V_P^2}>\nicefrac{V_A^2}{2k}$} \\

\end{cases} \]

\end{proposition}

\begin{proof}
Proof goes here
\end{proof}

The intuition for the result in \ref{prop2} is straightforward. When a
potential application is substantially more valuable (i.e. when
$\nicefrac{V_A^2}{2k}>V_P^2$), the platform decides to forgo monopoly
revenue from the platform itself to capturing additional revenue from
those who have a high valuation from the application. In this
case, everyone who buys the platform also buys the
application. 

Now we turn to analyzing the case where the potential application is
of the ``low'' type.


\begin{proposition}
\label{prop3}
When the potential application is of the low type, i.e. when
$V_A<V_P<\nicefrac{V_A}{k}$ in equilibrium, 
\begin{enumerate}

\item it is definitely the case
that some consumers buy the platform without buying applications.

\item the application cannot make positive revenues when
  $\frac{2V_A}{k+1}<V_P$

\item the platform always prices at $p=\frac{V_P}{2}$

\end{enumerate}
\end{proposition}

\begin{proof}
Proof goes here
\end{proof}



\section{Data}
test
\section{Conclusion}
test

\bibliographystyle{abbrv}
\bibliography{main}

\end{document}
This is never printed
}